%*******************************************************************************
%                                                                              *
%                 Datei: header.tex                                            *
%                                                                              *
%                 Stand: 05.06.2015   12.00 Uhr   (Brauer)                     *
%                                                                              *
%*******************************************************************************

\documentclass[%
	a4paper,			% Papierformat
	oneside,			% einseitiger Druck
	%twoside,			% zweiseitiger Druck
	12pt,				% Schriftgröße
	onecolumn,			% einspaltiger Text
	%twocolumn,			% zweispaltiger Text
	openright,			% Kapitel dürfen nur auf einer rechten Seite beginnen
	openany,			% Kapitel dürfen rechts oder links beginnen
	parskip=half,		% eine halbe Zeile Abstand zw. Absätzen
	headsepline,		% Kopfzeilenlinie
	footsepline,		% Fußzeilenlinie
	bibliography=totoc,	% Bibliographie im Inhaltsverzeichnis
	%idxtotoc			% Index im Inhaltsverzeichnis
	]{scrbook}
	
\usepackage[utf8]{inputenc}

\usepackage[
	left=30mm,
	right=20mm,
	top=45mm,
	bottom=55mm,
	%includeheadfoot,
	]{geometry}

% deutsche Silbentrennung etc.
\usepackage[ngerman]{babel}

% Grafiken: PDF, GIF, PNG
\usepackage{graphicx}

% Farben
\usepackage{color}
\definecolor{LinkColor}{rgb}{0,0,0.6}
\definecolor{ListingBG}{rgb}{0.9,0.9,0.9}

% Hyperlinks (anklickbar im PDF)
\usepackage[%
    pdftitle={Titel der Arbeit},%
    pdfauthor={Vorname Nachname},%
    pdfpagemode=UseOutlines
]{hyperref}   

\hypersetup{%
    colorlinks=true,%    farbige Links statt Rahmen
    linkcolor=LinkColor,
    citecolor=LinkColor,
    filecolor=LinkColor,
    menucolor=LinkColor,
    urlcolor=LinkColor,
    }

% Quellcode Formatierung
\usepackage{listings}
\lstset{%
    language=Perl, % Programmiersprache
    numbers=left, % Zeilennummern
    stepnumber=1, % jede Zeile
    numbersep=5pt, % Abstand zum Quellcode
    numberstyle=\tiny, % Schriftgröße
    breaklines=true, % Zeilenumbrüche zulassen
    breakautoindent=true, % Einrücken nach Umbruch
    tabsize=2,  % Tabulator
    basicstyle=\footnotesize, %
    showspaces=false, % Leerzeichen anzeigen (true -> underscore)
    showstringspaces=false, % Leerzeichen in Strings
    backgroundcolor=\color{ListingBG}, % Hintergrundfarbe
    captionpos=b,   % Position der Beschreibung (b: bottom)
    %keywordstyle=\color{red}\bfseries
}

% erweiterte Tabellen
\usepackage{array}

% Formelsatz
\usepackage{amsmath}

% Definition eigener Operatoren (im Header)
\DeclareMathOperator{\rg}{Rang}  

% Fortlaufende Kapitelüberschriften in der Kopfzeile
\pagestyle{headings}

\usepackage{scrpage2}
\pagestyle{scrheadings}
\setkomafont{pageheadfoot}{\normalfont\bfseries}
\renewcommand*\chapterpagestyle{scrheadings}
\renewcommand*\sectionmark[1]{\markright{\thesection\ #1}} 
\cfoot{}
\ofoot{\pagemark}

% Stil des Literaturverzeichnis
\bibliographystyle{alphadin}
\setbibpreamble{Beispielsweise Hinweis zur Sortierung des
Literaturverzeichnisses.\par\bigskip}
