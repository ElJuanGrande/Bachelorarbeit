\chapter{Einleitung}
Mit der fortschreitenden Verbreitung leistungsstarker Computersysteme im Consumer-Bereich und nicht zuletzt der Popularisierung mobiler Endgeräte, wie Smartphones und Tablets, rückte die Thematik der Usability immer stärker in den Fokus der Entwicklung von Anwendungssoftware. Besonders Websites und Anwendungen im privaten Sektor unterliegen diesem Trend. Im Bereich der Geschäftsanwendungen ist dieser Wandel nur schwach ausgeprägt. Nur wenige Geschäftsanwendungen hegen den Anspruch, eine ästhetische Bedienoberfläche bereitzustellen und intuitive Bedienkonzepte zu realisieren. Der dafür nötige Aufwand wird aus Kosten- oder Zeitgründen oft nicht unternommen. Dadurch entstehen starre Softwaremonolithen, die je nach Umfang eine stark anwachsende Lernkurve besitzen und umständlich zu handhaben sind.\par
Effiziente und gleichzeitig ansehnliche und leicht zu bedienende Softwareprodukte zu erstellen, sollte nicht nur im privaten Sektor von Interesse sein, sondern auch im betrieblichen. Der erforderliche Aufwand zahlt sich aus. Die Endanwender benötigen weniger Zeit und Hilfe, um sich in einer unbekannten Anwendung zurecht zu finden. Die Zufriedenheit und Produktivität der Anwender steigt, da sich schneller Erfolge beim Umgang mit der Software zeigen und die sich die Bedienung erfahrungsgemäß vertraut anfühlt.\par
Betrachtet man die jüngsten Entwicklungen, sollten moderne Anwendungen für den Einsatz auf mobilen Geräten, mit kleineren Bildschirmgrößen und Touchdisplays, im Unternehmensbereich vorbereitet sein und auf diesen ein vergleichbares Erlebnis liefern.\par
\section{Ziele der Arbeit} \label{sec:einlZiel}
Das Ziel dieser Arbeit besteht darin, eine Individualsoftware hinsichtlich der User-Experience zu optimieren. Dabei kommen den Aspekten Interaktionsdesign und Adaptive Design besondere Aufmerksamkeit zuteil. Unter Zuhilfenahme von Analyse- und Nutzertestmethoden werden Aussagen über Probleme bei der Bedienung der Anwendung getroffen, welche eine effiziente Ausführung von Aufgaben verhindern und ein positives Nutzerempfinden einschränken. Diese Probleme werden behoben, um eine Software zu erhalten, die sich an den Bedürfnissen der Nutzer orientiert. Die durchgeführten Analysen und entworfenen Konzepte sollen als Anregung für ähnliche Projekte dienen. Der Einblick in die Umsetzung mit dem JavaFX-Toolkit gibt Aufschluss über die technische Komponente der Ausarbeitung.\par
%\section{Motivation} \label{sec:einlMotivation}
\section{Aufbau der Arbeit} \label{sec:einlAufbau}
Zu Beginn der Ausarbeitung werden die Domäne und damit zusammenhängende Teilgebiete definiert und ein grundlegendes Wissen vermittelt, das für das Verständnis der Arbeit notwendig ist.\par
Zunächst werden dabei die Begriffe \textit{Usability} und \textit{User-Experience Design} vorgestellt und voneinander abgegrenzt. Im Rahmen des User-Experience Designs werden die Grundlagen des Interaktionsdesigns erläutert und die Möglichkeiten verschiedener möglicher Eingabegeräte eines Computersystems aufgezeigt. Es folgt die Vorstellung des Visual Designs, einer wichtigen Methodik im Prozess des Erstellens einer gebrauchstauglichen Anwendung. Psychologische Erkenntnisse zur menschlichen Wahrnehmung, die sogenannten Gestaltgesetze, unterstützen dabei, die Aufmerksamkeit des Nutzers auf bestimmte Oberflächenelemente zu lenken und den Zusammenhang zwischen diesen zu verdeutlichen. Standardisierte Richtlinien werden durch internationale Normen wie der DIN EN 9241 gegeben. Es werden verschiedene Methoden vorgestellt, die es mit verschieden großem Aufwand und Zuverlässigkeit ermöglichen, Aussagen über die Gebrauchstauglichkeit einer Anwendung zu treffen. Das in dem betrachteten Projekt verwendete UI-Toolkit \textit{JavaFX} wird in groben Zügen vorgestellt. Besonderer Fokus liegt auf der Logik des Eventhandlings, die im Laufe der Arbeit von Relevanz sein wird. Im Anschluss wird ein erster Einblick in die untersuchte Software gewährt und zuletzt kurz auf das Prinzip des Responsive Design eingegangen.\par
Das Kapitel 3 behandelt das Fachgebiet des Interaktionsdesigns. Zu Beginn wird das bereits vorhandene Interaktionskonzept der vorliegenden \textit{JavaFX}-Anwendung vorgestellt und analysiert. Auf Basis dieses Wissens wird überprüft, welche Bereiche der Anwendung auch mit der Tastatur gesteuert werden können. Für die Maussteuerung wird ein erweitertes Bedienkonzept entworfen, das sich an Touchgesten orientiert, wie sie häufig bei mobilen Geräten vorzufinden sind. Zu den einzelnen Problemen werden Lösungen entworfen und mit dem verwendeten UI-Toolkit \textit{JavaFX} realisiert.\par
Um die Individualisierbarkeit zu erhöhen und die Anwendung auf eine zukünftige Benutzung auf Tablets vorzubereiten, beschäftigt sich Kapitel 4 mit dem Entwurf und der Umsetzung einer Adaptive Design Lösung für die gesamte Anwendung, sodass die Benutzeroberfläche auf teils stark abweichenden Bildschirmgrößen noch immer ansehnlich dargestellt werden kann.\par
Die Kapitel 5 und 6 beschäftigen sich mit der Durchführung einiger Analysen und Nutzertests zur Feststellung der Gebrauchstauglichkeit der Software. Nach der Konzeption des Testaufbaus werden die verschiedenen Tests durchgeführt und die Ergebnisse ausgewertet. Für die gefundenen Probleme werden Lösungen erarbeitet und ebenfalls umgesetzt. Abschließend wird überprüft, ob die Änderungen das Problem lösen und zur Optimierung der User-Experience beitragen.\par
Im letzten Kapitel werden die gewonnenen Erkenntnisse und durchgeführten Tätigkeiten rekapituliert und ein wertendes Fazit gezogen.\par %Ausblick?