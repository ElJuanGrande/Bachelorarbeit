\chapter{Fazit und Ausblick}
\section{Zusammenfassung}
Im Rahmen dieser Arbeit wurde das Interaktionskonzept einer bestehenden modernen Desktopanwendung analysiert und erweitert. Die Erweiterung fand zunächst durch die Unterstützung von Tastatureingaben zur Steuerung der meisten Bedienelemente und Navigation zwischen den einzelnen Inhalten statt. Es wurden Konzepte entworfen, um gängige Gesten, die von mobilen Geräten bekannt sind, auf die Maussteuerung zu übertragen und so die Steuerung per Maus an die Kapazitäten von Touchdisplays anzugleichen. Gleichzeitig werden jene Gesten auch für die Bedienung mit einem Touchdisplay ermöglicht. Diese Konzepte wurden auf einzelne Komponenten oder komplexere darstellbare Inhalte angewandt und in Einzellösungen für die entsprechenden Elemente umgesetzt.\par
Mit dem Ziel, die Individualisierbarkeit der Anwendung zu erhöhen und die Lauffähigkeit auf sehr viel kleineren Bildschirmen, wie etwa einem Tablet-Display, zu ermöglichen, wurde analysiert, wodurch die Bedienbarkeit bislang eingeschränkt oder verhindert wurde. Zu diesen Problemen wurden Lösungen gefunden, die nach der Umsetzung eine kompaktere Darstellung der Anwendung erlaubten und die User-Experience dennoch nur geringfügig einschränken. Dabei wurde Rücksicht auf Designvorgaben genommen.\par
Unter Verwendung etablierter Usability- und User-Experience- Testverfahren, die sowohl durch ein heuristisches Verfahren, aber auch durch Nutzertests repräsentiert werden, wurde das Softwaresystem untersucht und Probleme in der Bedienung aufgedeckt. Unterstützt wurden die Nutzertests durch ein Eyetracking-Verfahren, das einen tiefergehenden Einblick in das Verhalten der Anwender bei der Bedienung der Anwendung gestattet. Auf Basis dieser Erkenntnisse war es möglich, die User-Experience im Rahmen des gegebenen Interaktionskonzeptes zu optimieren und erfahrenen sowie neuen Anwendern ein effizientes und zufriedenstellendes Bedienerlebnis zu ermöglichen.\par
\section{Fazit}
User-Experience ist ein spannendes und sehr aktuelles Thema, das sich in den letzten Jahren zurecht immer mehr Zuwendung erfreut. Besonders im Bereich der Geschäftsanwendungen ist es aus Kosten- oder Zeitgründen schwierig, ein ansprechendes, effizientes und intuitiv bedienbares Softwareprodukt zu finden. Durch ein gutes Interaktionsdesign, Visual Design und eine gute User-Experience, sticht ein solches System aus der Masse individueller Softwarelösungen hervor.\par
Innovative und intuitive Bedienkonzepte sind ein wichtiger Teil zum Erreichen eines zufriedenstellenden Ergebnisses. Mobile Geräte, wie Smartphones und Tablets, bieten einen exzellenten Anhaltspunkt für Konzepte, die auch auf andere Eingabegeräte übertragen werden können, da diese Geräte, und so auch die Bedienmöglichkeiten, mittlerweile Teil der täglichen Mensch-Computer-Interaktion vieler Personen sind. Dennoch muss bei der Umsetzung darauf geachtet werden, dass diese Möglichkeiten nur eine Erweiterung der normalen Interaktion darstellen und diese nicht ersetzen dürfen. Vor allem Consumer-Software, in der die User-Experience in der Regel weit stärker umgesetzt wird, kann als Orientierung für gewohnheitsgemäße Softwarebedienung dienen. Einige der verwendeten Methoden lassen sich durch ihren Wiedererkennungswert gut für Geschäftsanwendungen adaptieren. Ein Beispiel dafür ist die bei vielen Browsern verwendete Taste \textit{F11}, um den Vollbildmodus zu aktivieren. Dadurch, dass die Verwendung der meisten umgesetzten Funktionen zur intuitiven Bedienbarkeit optional ist, können erfahrene Benutzer des Systems eine hohe Effizienz erreichen, während sich neue Anwender zunächst auf einfache Eingabemethoden verlassen können.\par
Aus der Webentwicklung lässt sich viel über plattformunabhängige Darstellung lernen, auch wenn die dort verwendeten Methoden nur bedingt in einer Desktopanwendung übernommen werden können. Das moderne Responsive Design ist beispielsweise stark von den Sprachfeatures des HTML 5 und CSS 3 Standards abhängig. Zwar bietet das JavaFX-Toolkit eine gewisse CSS-Unterstützung um das Aussehen von Komponenten zu verändern, doch werden nur wenige Selektoren und ausgewählte Eigenschaften unterstützt. So bleibt nur eine Individuallösung für das Darstellen von Anwendungen auf Bildschirmen verschiedenster Größen.\par
Besonders interessant im Zuge der Usability-Tests ist das Eyetracking-Verfahren. Mittlerweile lassen sich bereits mit geringem Budget Hardware und Software beschaffen, die verschiedenste Analysen des Nutzerverhaltens ermöglichen. Die Ergebnisse, die dabei erzielt werden sind für die meisten Anwendungsfälle akkurat genug. Ob mit oder ohne Eyetracking-Unterstützung: Usability-Tests sind ein hervorragendes Mittel um Probleme bei der Bedienung eines Softwaresystems festzustellen. Für den größtmöglichen Nutzen sollten Usability-Tests allerdings in allen Phasen der Entwicklung regelmäßig durchgeführt werden.\par
Mit JavaFX wird den Entwicklern eine Technologie an die Hand gegeben, die der Umsetzung einer guten User-Experience nur wenige Grenzen setzt. Dennoch ist zu erwähnen, dass das Toolkit noch einige unausgereifte Funktionen und Fehler besitzt, welche die Programmierarbeit gelegentlich erschweren.\par
% Okay so? edit ?
Die gesteckten Ziele für die Optimierung der User-Experience konnten trotz der kleineren Schwierigkeiten mit der Oberflächenbibliothek erfüllt werden. Die Bedienung der Anwendung kann aufgrund der Verbesserungen auf intuitive und effizientere Weise erfolgen. Die Grenzen der Skalierbarkeit wurden, soweit es im Anforderungsrahmen möglich war, herabgesenkt. So ist es möglich, die Anwendung auf mobilen Geräten ohne eine nennenswerte Einschränkung der Gebrauchstauglichkeit zu verwenden. Die Kombination von Experten- und Nutzertests zur Feststellung von Usability-Problemen war sehr aufschlussreich und zeigte einige Probleme auf, die behandelt werden konnten. Die Verbesserungen führten im Gesamten zu einer optimierten User-Experience.\par
\section{Ausblick}
Bislang wurde die Anwendung durch die umgesetzten Verbesserungen im Bereich der User-Experience an die Bedienung auf kleinen Bildschirmen und die Steuerung Maus- und Touchgesten angepasst. Jedoch gibt es immernoch einige Funktionen, welche eine mobile Verwendung des Softwaresystems von der Bedienung auf einem Tablet oder Convertible unterscheidet. Darunter fallen unter anderem Tooltips und Hover-Effekte. Diese Arten visuellen Feedbacks sind auf Geräten mit Touchdisplay nicht verfügbar. Auch die Skalierbarkeit der Anwendung ist noch immer eingeschränkt. Für eine optimale User-Experience auf mobilen Geräten müssen einige Interaktionskonzepte überarbeitet werden. Ein Teil der Elemente, mit denen interagiert werden kann, wären aufgrund ihrer Größe auf mobilen Geräten nicht als solche erkennbar.\par
Auch im weiteren Projektverlauf können und sollten Usability-Tests durchgeführt werden, die sich auf neu entwickelte oder veränderte Funktionen beschränken. Dadurch kann die geschaffene User-Experience auch in Zukunft gewahrt bleiben.\par