\chapter{Adaptive Design}
Das Adaptive Design dient dazu, die Anwendung an verschiedene Bildschirmgrößen anzupassen. Die derzeit erforderliche Mindestauflösung der Monitore, um die Anwendung ohne Einschränkungen bedienen zu können, liegt bei 1680 x 1050 Pixel. Dieser Wert entspricht den kleinsten Arbeitsmonitoren der derzeitigen Endanwender. Durch Tests ist aufgefallen, dass bei Monitoren mit einer kleineren Auflösung, wie z.B. Tablet-PCs und Convertibles, einige Bedienelemente nicht mehr bedienbar sind. Sie werden außerhalb des sichtbaren Bereiches geschoben, ohne eine Möglichkeit zu diesen Komponenten zu scrollen, oder werden von anderen Objekten überlagert. Außerdem wird die Individualisierbarkeit der Anwednung durch die feste Größe des Fensters, die nicht unter 1680 x 1050 Pixel sinken kann, stark eingeschränkt.\par
Das Design muss daher an den entsprechenden Stellen so angepasst werden, dass die Größe der Anwendung variabler ist und diese auch auf Anzeigen mit geringerer Auflösung verwendet werden kann. Eine wichtige Anforderung ist, das Design in der optimalen Fenstergröße von 1680 x 1050 Pixeln (abzüglich Startleiste) nicht zu verändern.\par
\section{Konzept} \label{sec:responsiveConcept}
\editHere{Bilder}
Zunächst müssen die Problemstellen identifiziert werden. Für die Analyse wird die Beschränkung der Fenstergröße aufgehoben und die Anwendung in allen existierenden Ansichten verkleinert. Für die gefundenen Probleme werden Lösungen erarbeitet und später implementiert.\par
\heading{Hauptbildschirm}
Bereits bei der Analyse der Anwendung im Startbildschirm fällt auf, dass bei horizontaler Verkleinerung des Fensters die Sidebar aus dem rechten Fensterrand heraus verschwindet. Wird die Größe in vertikaler Ausrichtung verringert, repositioniert sich die Komponente im Content-Bereich nicht und überlagert ab einer gewissen Größe die Navigationsleiste. Sowohl die Seitenleiste als auch die Navigationsleiste werden durch diese Probleme zu großen Teilen unbedienbar. Innerhalb der Navigationsleiste wurden ebenfalls Probleme gefunden. Bei verringerter vertikaler Größe, überlagern nach einiger Zeit auch der Hilfetext und das \textit{FalkoFX}-Logo die Bedienelemente der Navigationsleiste, wodurch die darunter liegenden Elemente nicht mehr angeklickt werden können.\par
Zur Behebung der allgemeinen Positionierungsprobleme im Hauptfenster muss das Layoutverhalten des Wurzelknotens verändert werden, sodass die Seitenleiste eine unveränderliche Position hat und diese nicht mehr durch den Position und Größe des Hauptinhaltes bestimmt wird. Der Hauptinhalt sollte dabei ebenfalls eine variable Größe erhalten, die sich an dem verbleibenden Platz orientiert. Innerhalb der Navigationsleiste kann das Problem so behoben werden, dass der Hilfetext anfänglich die Breite je nach verbleibendem Platz verringert und bei zu geringer Breite verschwindet. Bietet der Navigationsbereich wieder ausreichend Platz, erscheint der Hilfetext wieder. Als zweite Stufe wird selbiges Verfahren auf das Logo angewandt, dass bei zu geringer Größe ebenfalls Elemente verbirgt.\par
\heading{Filterbildschirm}
Auch hier stellten sich zwei verschiedene Probleme heraus. Das erste Problem ist die Position des Radialmenüs. Dieses überlagert, sobald der vorhandene Platz nicht mehr ausreicht, die Navigationsleiste. Zudem stellt sich in dem Filter für den LTÜ-Anwendungsfall der Umstand ein, dass zwei Schaltflächen zum Umschalten des Radialmenüs in der linken oberen Ecke des Radialmenüs vorhanden sind. Auch diese werden durch das Radialmenü verdeckt.\par
Eine mögliche Lösung wäre, die Größe des Radialmenüs zu verringern, sobald der vorhandene Platz nicht mehr ausreicht. Dies widerspräche allerdings den Designrichtlinien, nach denen die Größe des runden Menüs exakt berechnet wurde, um im Einklang mit der Multi-Level-Liste und der Sidebar gesehen zu werden. Die andere Alternative ist es, das Verkleinern der Anwendung nur soweit zu erlauben, dass das Radialmenü noch in voller Größe (mit Abständen an allen Seiten) in dem Inhaltsbereich angezeigt werden kann. Dieser Wert liegt bei \editHere{X x Y Pixeln}. Das Verkleinern der Auswahlbuttons ist ebenfalls keine Option. Durch das Begrenzen der Mindestgröße der Anwendung allerdings wird nur die rechte der beiden Schaltflächen verdeckt. Kollidieren durch das Verkleinern des Fensters das Radialmenü und besagte Schaltfläche, kann die Position des Buttons verändert werden. Er würde dann in der rechten oberen Ecke erneut auftauchen. Ist wieder ausreichend Platz vorhanden, Kehrt die Schaltfläche an die ursprüngliche Position zurück. Unter normalen Umständen würde eine derartige Umsetzung nicht den Anforderungen genügen, da die Positionsänderung für leichte Verwirrung beim Anwender sorgen könnte, mangels Alternative muss die Umsetzung jedoch so erfolgen. Zudem ist sich der Nutzer bewusst, dass er bei starker Verkleinerung der Anwendung außerhalb der optimalen Benutzungsgröße arbeitet und nimmt das Risiko einer verringerten Usability im Austausch für die Individualisierbarkeit hin.\par
\heading{Ergebnisansichten}
Bei sämtlichen Ergebnisansichten fällt auf, dass diese sich bereits an die Größe des Inhaltsbereiches anpassen. In einigen Ansichten, wie der Ergebnistabelle und der Listenansicht wirken die Informationen bei kleineren Auflösungen jedoch stark gedrängt. Der Nutzer muss sehr viel scrollen, um in dem kleinen Bereich die Werte angezeigt zu bekommen, nach denen er im Augenblick sucht. Um dies angenehmer zu gestalten, sollte es möglich sein, die Sidebar und die Navigationsleiste übergangsweise auszublenden und so die Ergebnisansicht auf dem gesamten verfügbaren Raum präsentiert zu bekommen. Das Verlassen dieser Präsentation kann durch einen hinzugefügten Button oder durch das Drücken der ESC-Taste erfolgen. Um zu dem Verständnis der Funktion beizutragen, erfolgt der Zustandswechsel animiert, indem die Navigationsleiste nach oben aus de Fenster fährt und die Sidebar nach rechts. Währenddessen wird \enquote{wächst} der Inhaltsbereich auf die gesamte Fenstergröße an.
\section{Umsetzung} \label{sec:responsiveImplementation}
